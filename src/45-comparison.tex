\section{\FSCAFI{} semantic}

Since \Scafi{} is based on the notion of computations against a neighbour instead of neighbouring values it has different semantics from the higher order field calculus (\HFC) and its implementation Protelis. In order to formalize the semantic of \Scafi{} a subset of it has been considered called \textit{Featherweight \Scafi{} (\FSCAFI)} \cite{Scafi} and only the semantic of \FSCAFI has been considered. It has been proved that both \FSCAFI and \HFC{} can express programs that the other can't, but they have a non-empty intersection that can express all the self-stabilising build blocks.

\begin{figure}[t]
\centering
\centerline{$
\begin{array}{l@{\hspace{1mm}}c@{\hspace{1mm}}l@{\hspace{-8mm}}r}
        \PROGRAM & \BNFcce & \overline{\FUNCTION}  \; \e
                                                                                                                                                                                                        &   {\footnotesize \mbox{program}} \\[3pt]
        \FUNCTION & \BNFcce &  \defK \; \fname (\overline{\xname}) \; \eqSymK{\e}
                                                                                                                                                                                                        &   {\footnotesize \mbox{function declaration}} \\[3pt]
        \e & \BNFcce & \xname \, \BNFmid \, \anyvalue \, \BNFmid \, \e(\overline{\e}) \, \BNFmid \, \repK(\e)\{ \e \} \, \BNFmid \, \nbrK\{\e\} \, \BNFmid \, \foldK(\e)(\e)\{\e\} \,
                                                                                                                                                                                                        &   {\footnotesize \mbox{expression}} \\[3pt]
        \hline \\[-6pt]
        \anyvalue & \BNFcce &  {\dc(\overline{\anyvalue})} \; \BNFmid \; \funvalue
                                                                                                                                                                                                        &   {\footnotesize\mbox{value}} \\
        \funvalue & \BNFcce & \bname \; \BNFmid \; \fname \; \BNFmid \; (\overline{\xname}) \; \toSymKtag{\name}{\e}
                                                                                                                                                                                                        &   {\footnotesize\mbox{function value}} \\
\end{array}
$
}
\caption{Syntax of \FSCAFI{} from \cite{Scafi}}
\label{fig:fscafisyntax}
\end{figure}

The syntax of \FSCAFI is shown in figure \ref{fig:fscafisyntax}, it focus on the aggregate constructs ignoring the Scala specific features. A program $\PROGRAM$ is defined as a sequence of function declarations $\overline{\FUNCTION}$ followed by a main expression $\e$. A function declaration $\FUNCTION$ defines a (possibly recursive) function named $\fname$, with a sequence of formal parameters $\overline{\xname}$ and a body expression $\e$. An expression can be a variable $\xname$, a value $\anyvalue$, a function call $\e(\overline{\e})$ where $\e$ evaluates to a function $\funvalue$ and $\overline{\e}$ are the arguments, a $\mathtt{rep}$-expression $\repK(\e)\{ \e \}$ modelling time evolution, a $\mathtt{nbr}$-expression $\nbrK\{\e\}$ modelling neighbourhood interaction or a $\mathtt{foldhood}$-expression $\foldK(\e)(\e)\{\e\}$ combaining a computation against neighbours.

A value can be either a \textit{data value} $\dc(\overline{\anyvalue})$ consisting of a \textit{data constructor} $\dc$ applied to a sequence of values $\overline{\anyvalue}$, or a \textit{function value} $\funvalue$. A function value can be the name of a built-in function $\bname$, the name of a user defined function $\fname$, or an anonymous function $(\overline{\xname}) \; \toSymKtag{\name}{\e}$ where $\overline{\xname}$ are the formal parameters, $\e$ the body of the function and $\name$ a \textit{tag} that uniquely identifies the function and doesn't occours in the source program. A built-in function can be a \textit{pure operator} like the $\mathtt{mux}$ function or mathematical operators, a \textit{sensor} which depends on the environmental conditions of the computing device, or a \textit{relational sensor} which is a sensor depending additionally on a neighbouring device which the computation is happening against.

\begin{figure}[!t]{
 $\begin{array}{l}
\textbf{Value-trees  
and value-tree environments:}\\
\begin{array}{rcl@{\hspace{8.1cm}}r}
	\vtree & \BNFcce &  \mkvtree{E-Rule}{\anyvalue}{\overline{\vtree}}   
	&   {\footnotesize \mbox{value-tree}} \\
	\Trees & \BNFcce & \envmap{\overline{\deviceId}}{\overline{\vtree}} 
	 &   {\footnotesize \mbox{value-tree environment}} \\
\end{array}\\
\hline\\[-8pt]
\textbf{Auxiliary functions:}\\
\begin{array}{l}
\begin{array}{l@{\hspace{8pt}}l@{\hspace{8pt}}l}
	\nameOf((\overline{\xname}) \; \toSymKtag{\name}{\e}) = \name
	&
	\args{ (\overline{\xname}) \; \toSymKtag{\name}{\e}} = \overline{\xname}
	&
	\body{(\overline{\xname}) \; \toSymKtag{\name}{\e}} = \e
	\\
	\nameOf(\fname) = \fname
	&
	\args{\fname} = \overline{\xname}
	&
	\body{\fname} = \e  \; (\text{if } \, \defK \; \fname (\overline{\xname}) \; \eqSymK{\e})
	\\
	\nameOf(\bname) = \bname
	&
	\vrootOf{\mkvtree{}{\anyvalue}{\overline{\vtree}}}  =   \anyvalue
	\\
	\piIof{i}{\mkvtree{E-Rule}{\anyvalue}{\vtree_1,\ldots,\vtree_n}}  =  \vtree_i
	&
	\text{if~} 1\le i \le n
	&
	\text{~else~} \emptyseq
	\\
	\piBof{\funvalue}{\mkvtree{E-Rule}{\anyvalue}{\vtree_1,\ldots,\vtree_{n+2}}}  =   \vtree_{n+2}
	&
	\text{if~} \nameOf(\vrootOf{\vtree_{1}}) = \nameOf(\funvalue)
	&
	\text{~else~} \emptyseq
\end{array} \!\!\!\!\!\!\!\!\!\!\!\!\!\!\!
\\[30pt]
\mbox{For } \auxNAME\in\rho,\piI{i},\piB{\funvalue}:
\quad 
\left\{{\begin{array}{lcll}
	 \aux{\emptyseq}  & =  & \emptyseq
	\\
	\aux{\envmap{\deviceId}{\vtree}, \Trees}  & =   & \aux{\Trees}  & \quad \mbox{if} \; \aux{\vtree}=\emptyseq  
	\\
	 \aux{\envmap{\deviceId}{\vtree}, \Trees}  & =  & \envmap{\deviceId}{\aux{\vtree}}, \aux{\Trees} & \quad \mbox{if} \; \aux{\vtree} \not=\emptyseq  
\end{array}}\right.   \!\!\!\!\!\!
\end{array}\\
\hline\\[-10pt]
\textbf{Syntactic shorthands:}\\
\begin{array}{l@{\hspace{11pt}}l@{\hspace{11pt}}l}
\bsopsem{\deviceId,\deviceId'}{\piIofOv{\Trees}}{\senstate}{\overline{\e}}{\overline{\vtree}}
&
  \textrm{where~~} |\overline{\e}|=n
&
  \textrm{for~~}
  \bsopsem{\deviceId,\deviceId'}{\piIof{1}{\Trees}}{\senstate}{\e_1}{\vtree_1}
    \cdots
    \bsopsem{\deviceId,\deviceId'}{\piIof{n}{\Trees}}{\senstate}{\e_n}{\vtree_n} \!\!\!\!\!\!\!\!\!\!\!\! \\
\vrootOf{\overline{\vtree}}
&
  \textrm{where~~} |\overline{\vtree}|=n
  & \textrm{for~~}
\vrootOf{\vtree_1},\ldots,\vrootOf{\vtree_n}\\
\substitution{\overline{\xname}}{\vrootOf{\overline{\vtree}}}
&   \textrm{where~~} |\overline{\xname}|=n
  &
  \textrm{for~~}
\substitution{\xname_1}{\vrootOf{\vtree_1}}~\ldots\quad\substitution{\xname_n}{\vrootOf{\vtree_n}}
\end{array}\\
\hline\\[-10pt]
\textbf{Rules for expression evaluation:} \hfill
 \boxed{\bsopsem{\deviceId,\deviceId'}{\Trees}{\senstate}{\e}{\vtree}}
%
\skiptransitionR
%
\begin{array}{c}
\nullsurfaceTyping{E-VAL}{
\bsopsem{\deviceId,\deviceId'}{\Trees}{\senstate}{\anyvalue}{\mkvtree{}{\anyvalue}{}}
}
%
\skiptransitionN\\
%
\surfaceTyping{E-B-APP}{
\begin{array}{ll}
  \bsopsem{\deviceId,\deviceId}{\piIof{1}{\Trees}}{\senstate}{\e}{\vtree} 
                    &
          \bsopsem{\deviceId,\deviceId'}{\piIof{i+1}{\Trees}}{\senstate}{\e_i}{\vtree_i}  \quad \text{for all}\; i \in 1, \ldots, n
      \\
         \anyvalue=\builtinop{\bname}{\deviceId,\deviceId'}{\piBof{\bname}{\Trees},\senstate}(\vrootOf{\overline{\vtree}})
         &
         (\bname = \vrootOf{\vtree} \text{ is not relational }) \vee (\deviceId' \in \domof{\piBof{\bname}{\Trees}} \cup \{\deviceId\})
\end{array} \!\!\!\!
 }{
\bsopsem{\deviceId,\deviceId'}{\Trees}{\senstate}{\e(\overline{\e})}{\mkvtree{}{\anyvalue}{\vtree,\overline{\vtree},\anyvalue}}
}
%
\skiptransitionN\\
%
\surfaceTyping{E-D-APP}{
\begin{array}{ll}
  \bsopsem{\deviceId,\deviceId}{\piIof{1}{\Trees}}{\senstate}{\e}{\vtree} 
& 
\bsopsem{\deviceId,\deviceId'}{\piIof{i+1}{\Trees}}{\senstate}{\e_i}{\vtree_i} \quad \text{for all}\; i \in 1, \ldots, n
 
\\
  \funvalue = \vrootOf{\vtree} \mbox{ is not a built-in}
 & 
  \bsopsem{\deviceId,\deviceId'}{\piBof{\funvalue}{\Trees}}{\senstate}{\applySubstitution{\body{\funvalue}}{\substitution{\args{\funvalue}}{\vrootOf{\overline{\vtree}}}}}{\vtree'}
\end{array} \!\!\!\!
 }{
\bsopsem{\deviceId,\deviceId'}{\Trees}{\senstate}{\e(\overline{\e})}{\mkvtree{}{\vrootOf{\vtree'}}{\vtree,\overline{\vtree},\vtree'}}
}
%
\skiptransitionN\\
%
\surfaceTyping{E-REP}{
        \begin{array}{ll}
     \bsopsem{\deviceId,\deviceId}{\piIof{1}{\Trees}}{\senstate}{\e_1}{\vtree_1} & \anyvalue_1=\vrootOf{\vtree_{1}}\\
     \bsopsem{\deviceId,\deviceId}{\piIof{2}{\Trees}}{\senstate}{\e_2(\anyvalue_0)}{\vtree_2}~~& \anyvalue_2=\vrootOf{\vtree_{2}}
        \end{array}
        \quad
        \anyvalue_0 = \left\{\begin{array}{ll}
                             \vrootOf{\piIof{2}{\Trees}}(\deviceId) & \mbox{if} \;  \deviceId \in \domof{\Trees} \\
                             \anyvalue_1 & \mbox{otherwise}
                           \end{array}\right.
 }{
\bsopsem{\deviceId,\deviceId'}{\Trees}{\senstate}{\repK(\e_1)\{\e_2\}}{\mkvtree{}{\anyvalue_2}{\vtree_1,\vtree_2}}
}
%
\skiptransitionN\\
%
\surfaceTyping{E-NBR}{
         \quad
        \deviceId \neq \deviceId' \in \domof{\Trees}
\qquad
        \vtree = \Trees(\deviceId')
 }{
\bsopsem{\deviceId,\deviceId'}{\Trees}{\senstate}{\nbrK\{\e\}}{\vtree}
}
\qquad
\surfaceTyping{E-NBR-LOC}{
         \quad
     \bsopsem{\deviceId,\deviceId}{\piIof{1}{\Trees}}{\senstate}{\e}{\vtree}
 }{
\bsopsem{\deviceId,\deviceId}{\Trees}{\senstate}{\nbrK\{\e\}}{\mkvtree{}{\vrootOf{\vtree}}{\vtree}}
}
%
\skiptransitionN\\
%
\surfaceTyping{E-FOLD}{
        \begin{array}{ll}
                \bsopsem{\deviceId,\deviceId}{\piIof{1}{\Trees}}{\senstate}{\e_1}{\vtree^1}
                &
                 \quad
               \bsopsem{\deviceId,\deviceId}{\piIof{2}{\Trees}}{\senstate}{\e_2}{\vtree_0}
                  \qquad
               \funvalue = \vrootOf{\vtree_0}
                 \\
                 \deviceId_1,...\deviceId_n = \domof{\Trees}\cup\{\deviceId\} 
               &
               \quad
               n\ge m\ge 1
                \qquad \deviceId_1 = \deviceId
                \\
                 \bsopsem{\deviceId,\deviceId_i}{\piIof{3}{\Trees}}{\senstate}{\e_3}{\vtree_i}
                &
                \quad
               \text{for all } i \in 1,...,m 
                \\
                \bsopsemFAIL{\deviceId,\deviceId_j}{\piIof{3}{\Trees}}{\senstate}{\e_3}{}
                &
                 \quad
               \text{for all } j \in m+1,...,n 
                \\
                  \bsopsem{\deviceId,\deviceId}{\emptyset}{\senstate}{\funvalue(\vrootOf{\vtree^i}, \vrootOf{\vtree_i})}{\vtree^{i+1}}
                &
                \quad
                \text{for all } i \in 1,...,m
        \end{array}
 }{
\bsopsem{\deviceId,\deviceId'}{\Trees}{\senstate}{\foldK(\e_1)(\e_2)\{\e_3\}}{\mkvtree{}{\vrootOf{\vtree^{m+1}}}{\vtree^1, \vtree_0, \vtree_1}}
}
\end{array}
\end{array}$
} 
\vspace{-0.1cm}
 \caption{Big-step operational semantics for expression evaluation from \cite{Scafi}.} \label{fig:fscafisemantics}
\end{figure}

\begin{figure}[!t]{
 $\begin{array}{l}
\textbf{Auxiliary rules for expression evaluation failure:} \hfill
   %\qquad\qquad\qquad\quad\;\;
    \boxed{\bsopsemFAIL{\deviceId,\deviceId'}{\Trees}{\senstate}{\e}}
\skiptransition
\begin{array}{c}
\surfaceTyping{E-NBR-FAIL}{
    ~~ \quad
    \deviceId \neq \deviceId' \not\in \domof{\Trees}
 }{
	\bsopsemFAIL{\deviceId,\deviceId'}{\Trees}{\senstate}{\nbrK\{\e\}}
}
\skiptransitionN
\\
\surfaceTyping{E-R-APP-FAIL}{
	\begin{array}{ll}
	\bsopsem{\deviceId,\deviceId}{\piIof{1}{\Trees}}{\senstate}{\e}{\vtree} 
      &
      ~
       \bsopsem{\deviceId,\deviceId'}{\piIof{i+1}{\Trees}}{\senstate}{\e_i}{\vtree_i} ~~ \text{for all}\; i \in 1, \ldots, n
      \\
      \snvalue = \vrootOf{\vtree} \mbox{ is a relational-built-in}
      & 
    ~
    \deviceId \neq \deviceId' \not\in \domof{\piBof{\snvalue}{\Trees}}
	\end{array}
 }{
	\bsopsemFAIL{\deviceId,\deviceId'}{\Trees}{\senstate}{\e(\overline\e)}
}
\skiptransitionN
\\
%\surfaceTyping{E-APP-FUN-FAIL}{
%	~~ \quad
%	\bsopsemFAIL{\deviceId,\deviceId}{\piIof{n+1}{\Trees}}{\senstate}{\e}
% }{
%	\bsopsemFAIL{\deviceId,\deviceId'}{\Trees}{\senstate}{\e(\overline\e)}
%}
%\skiptransitionN
%\\
\surfaceTyping{E-APP-ARG-FAIL}{  
	\begin{array}{ll}
	 \bsopsem{\deviceId,\deviceId}{\piIof{1}{\Trees}}{\senstate}{\e}{\vtree}  
       & 
      \quad
       \overline{\e} = \e_1,...,\e_n 
        \qquad
         n\ge 0
	   \\
	  \bsopsem{\deviceId,\deviceId'}{\piIof{i+1}{\Trees}}{\senstate}{\e_i}{\vtree_i}
	&
	  \quad
	               \text{for all } i \in 1,...,m < n
	\\
	  \bsopsemFAIL{\deviceId,\deviceId'}{\piIof{m+2}{\Trees}}{\senstate}{\e_{m+1}}
	\end{array}
 }{
	\bsopsemFAIL{\deviceId,\deviceId'}{\Trees}{\senstate}{\e(\overline{\e})}
}
\skiptransitionN
\\
\surfaceTyping{E-D-APP-FAIL}{
	\begin{array}{ll}
		\bsopsem{\deviceId,\deviceId}{\piIof{1}{\Trees}}{\senstate}{\e}{\vtree} 
       & 
       \quad
       \bsopsem{\deviceId,\deviceId'}{\piIof{i+1}{\Trees}}{\senstate}{\e_i}{\vtree_i} ~~ \text{for all}\; i \in 1, \ldots, n
        \\
        \funvalue = \vrootOf{\vtree} \mbox{ is not a built-in}
        & 
        \quad
	  \bsopsemFAIL{\deviceId,\deviceId'}{\piBof{\funvalue}{\Trees}}{\senstate}{\applySubstitution{\body{\funvalue}}{\substitution{\args{\funvalue}}{\vrootOf{\overline{\vtree}}}}} %\!\!\!\!\!\!\!\!\!
	\end{array}
 }{
	\bsopsemFAIL{\deviceId,\deviceId'}{\Trees}{\senstate}{\e(\overline{\e})}
}
\end{array}
\end{array}$
} 
\vspace{-0.1cm}
\caption{Big-step operational semantics for expression evaluation from \cite{Scafi} (auxiliary rules for expression evaluation failure).} \label{fig:fscafisemanticsFAIL}
\end{figure}


The semantics of \FSCAFI is shown in figure \ref{fig:fscafisemantics}. Like the field calculus sematic it follows the same notation and the result of the execution is a value-tree. It also uses the same ausiliary functions. Unlike the field calculus semantics the expression are evaluated against an additional deviced id $\deviceId'$ which is used by the $\mathtt{foldhood}$-expressions to evaluate against a neighbour.

Rule \ruleNameSize{[E-VAL]} defines the evaluation of a value which produce a simple value-tree with only that value.

Rules \ruleNameSize{[E-B-APP]} and \ruleNameSize{[E-D-APP]} model the application of built-in and user-defined (or anonymous) functions. Like for the field calculus semantic the built-ins evaluation is delegated a specific ausiliary functions different for each built-in that may depend on the neighbour device against which the computation is currently happening. In case of user-defined functions the domain of aligned neighbours is restricted to only the ones that executed the same function thanks to the operator $\piBof{\funvalue}{\Trees}$.

Rule \ruleNameSize{[E-REP]}  models the evaluation of $\mathtt{rep}$-expressions just like the field calculus semantic.

Rules \ruleNameSize{[E-NBR]} and \ruleNameSize{[E-NBR-LOC]} both models the evaluation of $\mathtt{nbr}$-expressions, the former in the case that the computation is currently happening against a neightbour, the latter in case is happening against the executing device. In the first case the computed value is read from the value-tree environment, in the second case is computed recursively.

Rule \ruleNameSize{[E-FOLD]} models the evaluation of $\mathtt{foldhood}$-expressions. The expression $\e_3$ is computed against the executing device and against each network and the computed values are combined with the $\e_2$ function with an empty value-tree environment. For the evaluation of $\mathtt{foldhood}$ an additional set of rules, in figure \ref{fig:fscafisemanticsFAIL}, are required to handle the failures deriving by not aligned devices. The derive judgements for the failure cases are in the form $\bsopsemFAIL{\deviceId,\deviceId'}{\Trees}{\senstate}{\e}$ which means that expression $\e$ evaluation fails on the device $\deviceId$ against neightbour $\deviceId'$ with respect to the value-tree environment $\Trees$ and sensor state $\senstate$. The rule \ruleNameSize{[E-NBR-FAIL]} models the evaluation of an $\mathtt{nbr}$-expression against a non-aligned device, which results in a failure. Rule \ruleNameSize{[E-R-APP-FAIL]} models the failure on the evaluation of a built-in operators which is a relational sensor computing against a non-aligned device. Finally rules \ruleNameSize{[E-APP-ARG-FAIL]} and \ruleNameSize{[E-D-APP-FAIL]} model the propagation of failures when evaluating function application, the former handles failures in the evaluation of the arguments, the latter handles failure in the evaluation of the function.

\begin{figure}[!t]{
 $\begin{array}{l}
 
 \textbf{Network configurations and action labels:}\\
\begin{array}{lcl@{\hspace{7.6cm}}r}
%
\Field & \BNFcce &  \envmap{\overline\deviceId}{\overline\Trees}    &   {\footnotesize \mbox{computational field}} \\
\Topo & \BNFcce &  \envmap{\overline\deviceId}{\overline\devset}    &   {\footnotesize \mbox{topology}} \\
\Sens & \BNFcce &  \envmap{\overline\deviceId}{\overline\senstate}    &   {\footnotesize \mbox{sensors-map}} \\
\Envi & \BNFcce &  \EnviS{\Topo}{\Sens}    &   {\footnotesize \mbox{environment}} \\
\Cfg & \BNFcce &  \SystS{\Envi}{\Field}    &   {\footnotesize \mbox{network configuration}} \\
\act & \BNFcce &  \deviceId \;\BNFmid\; \envact    &   {\footnotesize \mbox{action label}} \\
%
\end{array}\\
\hline\\[-8pt]
\textbf{Environment well-formedness:}\\
\begin{array}{l}
%
\wfn{\EnviS{\Topo}{\Sens}} \textrm{~~holds if $\Topo,\Sens$ have same domain, and $\Topo$'s values do not escape it.}
\\
\end{array}\\
\hline\\[-8pt]
\textbf{Transition rules for network evolution:} \hfill
  \boxed{\nettran{\Cfg}{\act}{\Cfg}}
  \\[0.2cm]
\vspace{0.5cm}
\begin{array}{c}
\netopsemRule{N-FIR}{ \quad
                 \Envi=\EnviS{\Topo}{\Sens}
                   \quad \Topo(\deviceId)= \overline\deviceId 
                  \quad \bsopsem{\deviceId,\deviceId}{\filter_\deviceId(\Field)(\deviceId)}{\Sens(\deviceId)}{\emain}{\vtree}
                  \quad
                 \Field_1=\envmap{\overline\deviceId}{\{\envmap{\deviceId}{\vtree}\}}
                 }{
                 \nettran{\SystS{\Envi}{\Field}}{\deviceId}{\SystS{\Envi}{\mapupdate{\filter_\deviceId(\Field)}{\Field_1}}}
                 }
\skiptransition
\netopsemRule{N-ENV}
                 {\qquad \wfn{\Envi'}\qquad \Envi'=\EnviS{\Topo}{\envmap{\overline\deviceId}{\overline\senstate}} \qquad
                  
                  \Field_0=\envmap{\overline\deviceId}{\emptyset}
                 }
                 {\nettran{\SystS{\Envi}{\Field}}{\envact}{\SystS{\Envi'}{\mapupdate{\Field_0}{\Field}}}
                 }\\[-10pt]
\end{array}\\
%
%
%
\end{array}$
} 
 \caption{Small-step operational semantics for network evolution from \cite{Scafi}.} \label{fig:fscafinetworkSemantics}
\end{figure}

Figure \ref{fig:fscafinetworkSemantics} defines the operational semantics of whole networks. It uses the same syntax of the \HFC{} network semantics and the same rules. 
Rule \ruleNameSize{[N-FIR]} models a computational round on device $\deviceId$ by computing its local semantics on the filtered value-tree environment obtaining a value-tree $\vtree$, which is used to update the network configuration of the neighbours. Rule \ruleNameSize{[N-ENV]} models an environment change from $\Envi$ to a new well-formed environment $\Envi'$, the computational field is updated by removed the devices that are not part of the new environment and assigning an empty context to the new devices.

\section{\FSCAFI{} typing}

\begin{figure}[!t]{
 \scalebox{0.9}{
 $\begin{array}{l}
%%%  TYPES
\textbf{Types:}\\
\begin{array}{rcl@{\hspace{7.5cm}}r}
%
\type & \BNFcce &  \tvar  \; \BNFmid \;  \builtintype \; \BNFmid \;  (\overline\type) \rightarrow \type        &   {\footnotesize \mbox{type}} \\
%
\typescheme & \BNFcce &  \forall\overline{\tvar}.\type      &   {\footnotesize \mbox{type scheme}} \\
%
%\localenv & \BNFcce & \envS{\senstate}{\overline\vtree} &   {\footnotesize \mbox{local environment}} \\
\end{array}\\
\hline\\[-8pt]
%%%  TYPE RULES
\textbf{Expression typing:} 
  \hfill
  \boxed{\expTypJud{\TStypEnv}{\TtypEnv}{\e}{\type}}
\vspace{0.1cm}
  \\
\begin{array}{c}
%
\nullsurfaceTyping{T-VAR}{
\expTypJud{\TStypEnv}{\TtypEnv,\xname:\type}{\xname}{\type}
}
\qquad
%
{
\surfaceTyping{T-DAT}{ \quad
\applySubstitution{\type'}{\substitution{\overline\tvar}{\overline\type''}} = (\overline{\type})\rightarrow\type
\qquad
\expTypJud{\TStypEnv}{\TtypEnv}{\overline{\anyvalue}}{\overline{\type}}
}{
\expTypJud{\TStypEnv,\dc: \forall \overline{\tvar}. \type'}{\TtypEnv}{\dc(\overline{\anyvalue})}{\type} }
}
%
\skiptransition
%
\surfaceTyping{T-A-FUN}{ \quad
\expTypJud{\TStypEnv}{\;\TtypEnv,\,\overline{\xname}:\overline{\type}}{\e}{\type}
}{ \expTypJud{\TStypEnv}{\TtypEnv}{ (\overline{\xname}) \toSymKtag{\name}{\e}}{(\overline{\type})\rightarrow\type} }
%
\qquad
%
{
\surfaceTyping{T-N-FUN}{ \quad
\mbox{$\funvalue$ is a (built-in or declared) function}}{
\expTypJud{\TStypEnv,\funvalue: \forall \overline{\tvar}. \type}{\TtypEnv}{\funvalue}{\applySubstitution{\type}{\substitution{\overline\tvar}{\overline\type}}} }
}
%
\skiptransition
%
%
\surfaceTyping{T-APP}{ \quad
\expTypJud{\TStypEnv}{\TtypEnv}{\e}{(\overline{\type})\rightarrow\type} \qquad
\expTypJud{\TStypEnv}{\TtypEnv}{\overline{\e}}{\overline{\type}} }{
\expTypJud{\TStypEnv}{\TtypEnv}{\e(\overline{\e})}{\type} }
%
\skiptransition
%
%\surfaceTyping{T-BRANCH}{ \qquad
%\expTypJud{\TStypEnv}{\TtypEnv}{\e_1}{\btype}
%\quad \expTypJud{\TStypEnv}{\TtypEnv}{\e_2}{\type}
%\quad \expTypJud{\TStypEnv}{\TtypEnv}{\e_3}{\type} }{
%\expTypJud{\TStypEnv}{\TtypEnv}{\ifK(\e_1) \{\e_2\} \{\e_3\} }{\type} }
%%
%\skiptransition
%%
\surfaceTyping{T-REP}{ \qquad
\expTypJud{\TStypEnv}{\TtypEnv}{\e_1}{\type}
\qquad \expTypJud{\TStypEnv}{\TtypEnv}{\e_2}{(\type)\to \type} }{
\expTypJud{\TStypEnv}{\TtypEnv}{\repK(\e_1)\{\e_2\}}{\type} }
%
\qquad\qquad
%
\surfaceTyping{T-NBR}{ \qquad
\expTypJud{\TStypEnv}{\TtypEnv}{\e}{\type}
}{ \expTypJud{\TStypEnv}{\TtypEnv}{\nbrK\{\e\}}{\type} }
%
\skiptransition
%
\surfaceTyping{T-FOLD}{ \qquad
\expTypJud{\TStypEnv}{\TtypEnv}{\e_1}{\type}
\quad \expTypJud{\TStypEnv}{\TtypEnv}{\e_2}{(\type, \type) \to \type}
\quad \expTypJud{\TStypEnv}{\TtypEnv}{\e_3}{\type} }{
\expTypJud{\TStypEnv}{\TtypEnv}{\foldK(\e_1) (\e_2) \{\e_3\} }{\type} }
%
\skiptransition
%
\end{array}
\\
\textbf{Function typing:} 
  \hfill
  \boxed{\funTypJud{\TStypEnv}{\FUNCTION}{\typescheme}}
  \\
\begin{array}{c}
%
\surfaceTyping{T-FUNCTION}{
\qquad
\expTypJud{\TStypEnv,\,\fname:\forall\emptyseq.(\overline{\type})\rightarrow\type}{\overline{\xname}:\overline{\type}}{\e}{\type}
\qquad
\overline{\tvar}=\FTV{(\overline{\type})\rightarrow\type}
}{ \funTypJud{\TStypEnv}{\defK \; \fname (\overline{\xname}) \; \eqSymK{\e}}{\forall\overline{\tvar}.(\overline{\type})\rightarrow\type}}
%
\skiptransition
%
\end{array}
\\
\textbf{Program typing:} 
  \hfill
  \boxed{\proTypJud{\PROGRAM}{\type}}
  \\
\begin{array}{c}
%
\surfaceTyping{T-PROGRAM}{
\\
\TStypEnv_0=\OStypEnv
\\
\FUNCTION_i = \defK \; \fname_i (\_) \; \eqSymK{\_}
\qquad
\funTypJud{\TStypEnv_{i-1}}{\FUNCTION_i}{\typescheme_i}
\qquad
\TStypEnv_i=\TStypEnv_{i-1},\, \fname_i:\typescheme_i
\qquad
 (i \in 1..n)
\\
\expTypJud{\TStypEnv_n}{\emptyset}{\e}{\type}
}{ \proTypJud{\FUNCTION_1\cdots\FUNCTION_n  \;
\e}{\type}}
\end{array}
\end{array}$}
} \caption{Type rules for expressions, function declarations, and programs. from \cite{Scafi}} \label{fig:fscafiTyping}
\end{figure}

Figure \ref{fig:fscafiTyping} present a type system for \FSCAFI from \cite{Scafi}. A type $\type$ can be either a type variable $\tvar$, a built-in type $\builtintype$ or a function type $(\overline\type) \rightarrow \type$. Type schemes $\typescheme$ can produce regulart types $\type$ by replacing the type variables with a type. The typing jedgement rules are given in the form $\expTypJud{\TStypEnv}{\TtypEnv}{\e}{\type}$, meaning that $\e$ has type $\type$ under the type-scheme assumptions $\TStypEnv$ and the type assumtions $\TtypEnv$. A type scheme environement $\TStypEnv$ collect the type schemes for built-in constructors, built-in operators and user-defined functions. A type environment $\TtypEnv$ collects the types for program variables. $\OStypEnv$ denotes the distinguished built-in type-scheme environment for built-in constructors and built-in functions. $\FTV{\type}$ denotes the set of type variables occurring in the type $\type$.

Rule \ruleNameSize{[T-VAR]} assigns a type to a variable $\xname$ by looking the type assumptions $\TtypEnv$. Rule \ruleNameSize{[T-DAT]} assigns a type to a data value by extracting the type scheme of the data constructor from the type-scheme environment $\TStypEnv$. Rules \ruleNameSize{[T-A-FUN]} and \ruleNameSize{[T-N-FUN]} assign a type to anonymous function and built-in or declared function names, respectively. Rule \ruleNameSize{[T-APP]} assigns a type to a function application. Rule \ruleNameSize{[T-REP]} assigns a type to a $\mathtt{rep}$-expression, ensuring that the initial value and the domain and range of the body have the same type. Rule \ruleNameSize{[T-NBR]} assings a type to a $\mathtt{nbr}$-expression that has the same type as the expression inside the operator. Rule \ruleNameSize{[T-FOLD]} assigns a type to a $\mathtt{foldhood}$-expression, ensuring that $\e_1$ and $\e_3$ have the same type $\type$ and that $\e_2$ has type $(\type,\type) \to \type$.

Rule \ruleNameSize{[T-FUNCTION]} assigns a type scheme to a user defined function. Rule \ruleNameSize{[M-PROGRAM]} allows to type a program with the type of the main expression by adding to the type-scheme environment the type schemes of all the user defined functions.

\section{\Scafi{} and HFC semantics alignment}

\begin{figure}[!t]
\centering
\centerline{$
\begin{array}{ccc}
\HFC{} && \FSCAFI{} \\
\defK \; \fname (\overline{\xname}) \; \{\e\} &\longleftrightarrow& \defK \; \fname (\overline{\xname}) \; \eqSymK{\e} \\
(\overline{\xname}) \; \toSym{\name} \; \e &\longleftrightarrow& (\overline{\xname}) \; \toSymKtag{\name}{\e} \\
\foldK(\e, \e, \e) &\longleftrightarrow& \foldK(\e)(\e)\{\e\}
\end{array}
$
}
\caption{Informal description of the bidirectional translation between \HFC{} and \FSCAFI{} from \cite{Scafi}.} \label{fig:fscafitranslation}
\end{figure}

A \FSCAFI program can be translated into a \HFC{} program and viceversa following simple translation rules shown in figure \ref{fig:fscafitranslation}, assuming the existence of a $\foldK$ built-in function for HFC equivalent to the $\foldK$ operator of \FSCAFI. Those rules don't preserve type-safety and behavior for all programs, however they are preserved for a fragment of the two languages, called respectively HFC' and \textit{Aligned} \FSCAFI. HFC' is obtained by applying three restriction on how field values can be processed:
\begin{description}
	\item[R1]
	Expressions of neighbouring type can only be aggregated to local values with a $\foldK$ operator if they do not capture variables of neighbouring types; so that, e.g., aggregating arguments of neighbouring type is never allowed.
	\item[R2]
	Functions $\funvalue$ with arguments of neighbouring type have to return a neighbouring type.
	\item[R3]
	Built-in functions need to be pointwise or aggregating on neighbouring values.
\end{description}
The reason behind the first rule are visible in the following example:
\begin{lstlisting}[]
def wrong_avghood(x) = {
  foldhood(0, +, x) / foldhood(0, +, 1)
}
wrong_avghood(nbr{sns-temp()})
\end{lstlisting}
and its translation in  \FSCAFI:
\begin{lstlisting}[]
def wrong_avghood(x) = @@{
  foldhood(0)(+){x} / foldhood(0)(+){1}
}
wrong_avghood(nbr{sns-temp()})
\end{lstlisting}
The \HFC{} program computes the average temperature of neighbours while the \FSCAFI{} program is equivalent to the program $\nbrK\{\texttt{sns-temp}()\}$. In the \FSCAFI program if the computation is happening against a neighbour $\deviceId'$ only that temperature $\mathtt{t}$ is used as argument of the function, then it simply compute $nt/n = t$ where $n$ is the number of neighbours. In \Scafi{} a simple solution to write this program is to make $\xname$ a by-name parameter.

The reason for the the second restriction is exemplified in the following program, in which if and branch are both implemented in terms of the $\mathtt{mux}$ built-in function:
\begin{lstlisting}[]
def wrong_ignore(x) = { 1 }
foldhood(0, +, if (sns-temp() > 0) { wrong_ignore(nbr{sns-temp()}) } { 1 } )
\end{lstlisting}
and its translation in  \FSCAFI{}
\begin{lstlisting}[]
def wrong_ignore(x) = @@{ 1 }
foldhood(0)(+){ branch (sns-temp() > 0) { wrong_ignore(nbr{sns-temp()}) } { 1 } }
\end{lstlisting}
The \HFC{} program computes the number of neighbours while the \FSCAFI{} program in case the executing device has a positive temperatures counts only the number of neighbour with positive temperature since the fooldhood evaluation fail against the others.

The reason behind the third rule is exaplained by considering for example a buil-in function $\mathtt{sorthood}$ which rearranges values $\fvalue(\deviceId)$ in increasing order of neighbour identifiers $\deviceId$. So given a neighbouring value $\fvalue = \envmap{\overline\deviceId}{\overline\lvalue}$ and assuming $\deviceId_1 \leq \ldots \leq \deviceId_n$ it returns a neighbouring value $\fvalue' = \envmap{\deviceId_1}{\lvalue_{\pi_1}}, \ldots, \envmap{\deviceId_n}{\lvalue_{\pi_n}}$. this function is conceivable in \HFC{} but it's impossible to implement in \FSCAFI{}.

Those restriction rules can be formalized in \FSCAFI{} by a new set of more restrictive typing rules, obtaining a fragment of the language called \textit{Aligned} \FSCAFI{}. It can be proved \cite{Scafi} that a well-typed program of one of the fragments can be transformed into a well-type program of the other with the same behavior. It can also be proved that Aligned \FSCAFI{} is Turing universal and all the base programming patterns provided in \cite{SelfStabilizing} for designing self-stabilising applications can be rewritten to fit into the fragment.

\begin{figure}[t!]
\scalebox{0.9}{$\begin{array}{l}
%%%  TYPES
\begin{array}{r@{\hspace{1.0cm}}c}
	\begin{array}{rcl@{\hspace{1cm}}r}
		\multicolumn{4}{l}{\textbf{Types:}} \\[1pt]
		\type	& \BNFcce & \tvar \; \BNFmid \; {\rtype} \; \BNFmid \; \ltype											   &   {\footnotesize \mbox{type}} \\
		\ltype	& \BNFcce & \ltvar \; \BNFmid \; {\stype} \; \BNFmid \; (\overline{\type})\to\ftype	 & {\footnotesize \mbox{local type}} \\
		\rtype	& \BNFcce & \rtvar \; \BNFmid \; \stype \; \BNFmid \; \ftype										  &   {\footnotesize \mbox{return type}} \\
		\stype	& \BNFcce & \stvar \; \BNFmid \; \builtintype \; \BNFmid \; (\overline{\ltype})\to\stype	&   {\footnotesize \mbox{local return type}} \\
		\ftype	& \BNFcce & \ftypeOf{\stype}																							&   {\footnotesize \mbox{field type}} \\[2pt]
		\multicolumn{4}{l}{\textbf{Local type schemes:}} \\[1pt]
		\ltypescheme & \BNFcce & \forall\overline{\tvar}\overline{\ltvar}\overline{\rtvar}\overline{\stvar}.\ltype
																																							  &   {\footnotesize \mbox{local type scheme}} \\
	\end{array}
	&
	{\raisebox{-1.3cm}{
	\begin{tikzpicture}[xscale=1.5,yscale=1.1]
		\draw[ultra thick] (-0.5,0) circle (1cm);
		\draw[ultra thick] (0.5,0) circle (1cm);
		\node[draw,rectangle] (TT) at (0, 1.25) {$\type$};
		\node[draw,rectangle] (LT) at (-1, 0.5) {$\ltype$};
		\node[draw,rectangle] (RT) at (1, 0.5) {$\rtype$};
		\node[draw,rectangle] (ST) at (0, -0.5) {$\stype$};
		\node (B) at (0.1, 0.5) {$\builtintype$};
		\node (TxS) at (0, 0.05) {$(\overline\ltype) \rightarrow \stype$};
		\node (F) at (0.8, -0.4) {$\ftype$};
		\node (TxF) at (-0.95, -0.3) {$(\overline\type) \rightarrow \ftype$};
	\end{tikzpicture}
	}}
\end{array} \\[38pt]
\hline\\[-8pt]
%%%  EXPRESSIONS
\textbf{~~Expression typing:} \hfill \boxed{\expTypJud{\LTStypEnv}{\TtypEnv}{\e}{\type}} \\[5pt]
\quad
\begin{array}{c}
	\nullsurfaceTyping{T'-VAR}{
		\expTypJud{\LTStypEnv}{\TtypEnv, \xname:\type}{\xname}{\type}
	}
	\qquad
	\surfaceTyping{T'-DAT}{ \quad
		\applySubstitution{\stype'}{\substitution{\overline\stvar}{\overline\stype''}} = (\overline{\stype})\rightarrow\stype
		\qquad
		\expTypJud{\TStypEnv}{\TtypEnv}{\overline{\lvalue}}{\overline{\stype}}
	}{
		\expTypJud{\TStypEnv,\dc: \forall\overline\stvar. \stype'}{\TtypEnv}{\dc(\overline{\lvalue})}{\stype}
	}
	\skiptransition
	\surfaceTyping{T'-A-FUN}{  \quad
		\overline{\yname}=\FV{(\overline{\xname}) \; \toSymKtag{\name}{\e}} \quad
		{\expTypJud{\LTStypEnv}{\TtypEnv}{\overline\yname}{\overline{\ltype}}} \quad
		\expTypJud{\LTStypEnv}{\TtypEnv, \overline{\xname} : \overline\type}{\e}{\rtype}
	}{
		\expTypJud{\LTStypEnv}{\TtypEnv}{(\overline{\xname}) \; \toSymKtag{\name}{\e}}{(\overline\type) \rightarrow \rtype}
	}
	\skiptransition
	\surfaceTyping{T'-N-FUN}{ \quad
		\mbox{$\funvalue$ is a (built-in or declared) function}
	}{
		\expTypJud{\TStypEnv,\funvalue : \forall\overline\tvar\overline\ltvar\overline\rtvar\overline\stvar.\ltype}{\TtypEnv}{\funvalue}{\applySubstitution{\ltype}{\substitution{\overline\tvar}{\overline\type},~ \substitution{\overline{\ltvar}}{\overline{\ltype}},~ \substitution{\overline{\rtvar}}{\overline{\rtype}},~ \substitution{\overline{\stvar}}{\overline{\stype}}}}
	}
	\skiptransition
	\surfaceTyping{T'-APP}{ \qquad
		\expTypJud{\LTStypEnv}{\TtypEnv}{\e}{(\overline\type)\rightarrow\rtype} \qquad
		\expTypJud{\LTStypEnv}{\TtypEnv}{\overline{\e}}{\overline\type}
	}{
		\expTypJud{\LTStypEnv}{\TtypEnv}{\e(\overline{\e})}{\rtype}
	}
	\skiptransition
	\surfaceTyping{T'-REP}{ \quad
		\expTypJud{\LTStypEnv}{\TtypEnv}{\e_1}{\stype} \qquad
		\expTypJud{\LTStypEnv}{\TtypEnv}{\e_2}{\stype \rightarrow \stype}
	}{
		\expTypJud{\LTStypEnv}{\TtypEnv}{\repK(\e_1)\{\e_2\}}{\stype}
	}
	\qquad
	\surfaceTyping{T'-NBR}{ \quad
		\expTypJud{\LTStypEnv}{\TtypEnv}{\e}{\stype}
	}{
		\expTypJud{\LTStypEnv}{\TtypEnv}{\nbrK\{\e\}}{\ftypeOf{\stype}}
	}
	\skiptransition
	\surfaceTyping{T'-FOLD}{ \qquad
		\begin{array}{ll}
			\expTypJud{\TStypEnv}{\TtypEnv}{\e_1}{\stype}
			& \quad 
			\expTypJud{\TStypEnv}{\TtypEnv}{\e_3}{\ftypeOf{\stype} \text{ or } \stype}
			\\
			\expTypJud{\TStypEnv}{\TtypEnv}{\e_2}{(\stype, \stype) \rightarrow \stype}
			& \quad 
	        \expTypJud{\TStypEnv}{\TtypEnv}{\overline\xname}{\overline\ltype}
	        \quad\text{where}\; \overline\xname = \FV{\e_3}
		\end{array}
	}{
		\expTypJud{\TStypEnv}{\TtypEnv}{\foldK(\e_1) (\e_2) \{\e_3\} }{\stype}
	}
\end{array} \\~\\
%%%  FUNCTIONS
\textbf{~~Function typing:} \hfill \boxed{\funTypJud{\LTStypEnv}{\FUNCTION}{\ltypescheme}} \\[5pt]
\qquad \qquad
\begin{array}{c}
	\surfaceTyping{T'-FUNCTION}{ \qquad
		\begin{array}{ll}
			\expTypJud{\LTStypEnv, \fname: (\overline\type)\to\rtype}{\overline{\xname}:\overline{\type}}{\e}{\rtype} \qquad
			\overline{\tvar}\overline{\ltvar}\overline{\rtvar}\overline{\stvar}=\FTV{(\overline{\type})\rightarrow\rtype} %\\
			%\text{arguments not occurring in } \e \text{ are local}
		\end{array}
	}{
		\funTypJud{\LTStypEnv}{\defK \; \fname (\overline{\xname}) \; \eqSymK{\e}}{
			\forall\overline{\tvar}\overline{\ltvar}\overline{\rtvar}\overline{\stvar}.(\overline{\type})\rightarrow\rtype
	}}
\end{array} \\~\\
%%%  PROGRAMS
\textbf{~~Program typing:} \hfill {\boxed{\funTypJud{\LTStypEnv_0}{\PROGRAM}{\type}}} \\[5pt]
\qquad \quad
\begin{array}{c}
	\surfaceTyping{T'-PROGRAM}{ \\
		\FUNCTION_i=(\defK \; \fname_i (\_) \; \_) \qquad
		\funTypJud{\LTStypEnv_{i-1}}{\FUNCTION_i}{\ltypescheme_i} \qquad
		\LTStypEnv_i=\LTStypEnv_{i-1},\, \fname_i:\ltypescheme_i \qquad
		(i \in 1..n) \\
		\expTypJud{\LTStypEnv_n}{\emptyset}{\e}{\type}
	}{
		\funTypJud{{\LTStypEnv_0}}{\FUNCTION_1\cdots\FUNCTION_n  \; \e}{{\type}}
	}
\end{array}
\end{array}$}
\caption{Hindley-Milner typing for Aligned \FSCAFI{} expressions, function declarations, and programs. from \cite{Scafi}} \label{fig:alignfscafityping}
\end{figure}

Figure \ref{fig:alignfscafityping} shows the typing rules of aligned \FSCAFI. The types has been divided in four categories, of which only one can appear as a field type from an $\mathtt{nbr}$-expression. The restriction R1 is further implemented by requiring in the rule \ruleNameSize{[T-FOLD]} that each free variable occurring in $\e_3$ has a local type. The restriction R2 is implemented by restricting local function types to $(\overline{\type})\to\ftype$ instead of $(\overline{\type})\to\rtype$. Restriction R3 is implicitly valid for all \FSCAFI programs given the semantic rules.
